\section{Question 3: Challenges} 

It was a challenge to come together as a time and coordinate our schedules. Another challenge was trying to distill the information shared in class and parse out what was truly necessary in order to get the car to run in the way we desired versus what was auxiliary information that was intended for our broader understanding of autonomous robots. 

Furthermore, it was a challenge to figure out the branches and quadrants that we needed to reason over in order to properly properly compute the angle (in radians) needed to compute the free path length. Finally, it was a challenge to properly tune the reward function. 

We overcame the scheduling issues by being comfortable with the fact that there may be times when only two of us could be there, and we always trusted the third to put in work another time. We overcame the knowledge distillation through open and candid conversations amongst ourselves. The mathematical challenges were overcome by stepping away from our computers and working it out on a whiteboard. Finally, the challenges of tuning were solved by just running it over and over whether in the real world, in simulation, or by joysticking the robot to best intuit the solution and hyper-parameters. 